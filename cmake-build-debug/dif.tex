\documentclass{article}
\usepackage[utf8]{inputenc}
\usepackage[russian]{babel}
\usepackage{graphicx}
\usepackage{environ}
\usepackage{etoolbox}
\usepackage{amsmath}
\usepackage{resizegather}

\begin{document}

Требуется разложить в ряд Тейлора до $o({(x+1)}^{7})$
\begin{gather}\sin{ \left(x + y\right) }
\end{gather}Значение функции в заданной точке: $0$ \\ 
Вычисляем
\begin{gather}
\frac{\partial ^{1}}{\partial {x}^{1}}\left(\sin{ \left(x + y\right) }\right)=\frac{\partial }{\partial x}\left(\sin{ \left(x + y\right) }\right)
\end{gather}
Смотрим в книгу, видим
\begin{gather}
\frac{\partial }{\partial x}\left(\left(x + y\right)\right)=1
\end{gather}
Дальнейший переход является упражнением для читателя
\begin{gather}
\frac{\partial }{\partial x}\left(\sin{ \left(x + y\right) }\right)=\cos{ \left(x + y\right) }
\end{gather}
Значение производной в заданной точке: $1$ \ 
Вычисляем
\begin{gather}
\frac{\partial ^{2}}{\partial {x}^{2}}\left(\sin{ \left(x + y\right) }\right)=\frac{\partial }{\partial x}\left(\cos{ \left(x + y\right) }\right)
\end{gather}
Как говорил Георг Кантор, сущность математики - в ее свободе. Без замедлений применим все необходимые правила
\begin{gather}
\frac{\partial }{\partial x}\left(\left(x + y\right)\right)=1
\end{gather}
Ясно как божий день, что
\begin{gather}
\frac{\partial }{\partial x}\left(\cos{ \left(x + y\right) }\right)=(-1) \cdot \sin{ \left(x + y\right) }
\end{gather}
Значение производной в заданной точке: $-0$ \ 
Вычисляем
\begin{gather}
\frac{\partial ^{3}}{\partial {x}^{3}}\left(\sin{ \left(x + y\right) }\right)=\frac{\partial }{\partial x}\left((-1) \cdot \sin{ \left(x + y\right) }\right)
\end{gather}
Очевидно, что
\begin{gather}
\frac{\partial }{\partial x}\left(\left(x + y\right)\right)=1
\end{gather}
Нетрудно получить
\begin{gather}
\frac{\partial }{\partial x}\left(\sin{ \left(x + y\right) }\right)=\cos{ \left(x + y\right) }
\end{gather}
Данное рассказывают еще в детском саду:
\begin{gather}
\frac{\partial }{\partial x}\left((-1) \cdot \sin{ \left(x + y\right) }\right)=\cos{ \left(x + y\right) } \cdot (-1)
\end{gather}
Значение производной в заданной точке: $-1$ \ 
Вычисляем
\begin{gather}
\frac{\partial ^{4}}{\partial {x}^{4}}\left(\sin{ \left(x + y\right) }\right)=\frac{\partial }{\partial x}\left(\cos{ \left(x + y\right) } \cdot (-1)\right)
\end{gather}
Дальнейший переход является упражнением для читателя
\begin{gather}
\frac{\partial }{\partial x}\left(\left(x + y\right)\right)=1
\end{gather}
Таким образом,
\begin{gather}
\frac{\partial }{\partial x}\left(\cos{ \left(x + y\right) }\right)=(-1) \cdot \sin{ \left(x + y\right) }
\end{gather}
Следовательно
\begin{gather}
\frac{\partial }{\partial x}\left(\cos{ \left(x + y\right) } \cdot (-1)\right)=\sin{ \left(x + y\right) }
\end{gather}
Значение производной в заданной точке: $0$ \ 
Вычисляем
\begin{gather}
\frac{\partial ^{5}}{\partial {x}^{5}}\left(\sin{ \left(x + y\right) }\right)=\frac{\partial }{\partial x}\left(\sin{ \left(x + y\right) }\right)
\end{gather}
Далее
\begin{gather}
\frac{\partial }{\partial x}\left(\left(x + y\right)\right)=1
\end{gather}
Ясно как божий день, что
\begin{gather}
\frac{\partial }{\partial x}\left(\sin{ \left(x + y\right) }\right)=\cos{ \left(x + y\right) }
\end{gather}
Значение производной в заданной точке: $1$ \ 
Вычисляем
\begin{gather}
\frac{\partial ^{6}}{\partial {x}^{6}}\left(\sin{ \left(x + y\right) }\right)=\frac{\partial }{\partial x}\left(\cos{ \left(x + y\right) }\right)
\end{gather}
Ясно как божий день, что
\begin{gather}
\frac{\partial }{\partial x}\left(\left(x + y\right)\right)=1
\end{gather}
Смотрим в книгу, видим
\begin{gather}
\frac{\partial }{\partial x}\left(\cos{ \left(x + y\right) }\right)=(-1) \cdot \sin{ \left(x + y\right) }
\end{gather}
Значение производной в заданной точке: $-0$ \ 
Вычисляем
\begin{gather}
\frac{\partial ^{7}}{\partial {x}^{7}}\left(\sin{ \left(x + y\right) }\right)=\frac{\partial }{\partial x}\left((-1) \cdot \sin{ \left(x + y\right) }\right)
\end{gather}
Таким образом,
\begin{gather}
\frac{\partial }{\partial x}\left(\left(x + y\right)\right)=1
\end{gather}
Посредством пристального вглядывания в выражение выводим
\begin{gather}
\frac{\partial }{\partial x}\left(\sin{ \left(x + y\right) }\right)=\cos{ \left(x + y\right) }
\end{gather}
Смотрим в книгу, видим
\begin{gather}
\frac{\partial }{\partial x}\left((-1) \cdot \sin{ \left(x + y\right) }\right)=\cos{ \left(x + y\right) } \cdot (-1)
\end{gather}
Значение производной в заданной точке: $-1$ \ 
Искомое разложение:
\begin{gather}
\sin{ \left(x + y\right) }=\frac{{(x+1)}^{1}}{1!}-\frac{{(x+1)}^{3}}{3!}+\frac{{(x+1)}^{5}}{5!}-\frac{{(x+1)}^{7}}{7!}+o({(x+1)}^{7})\end{gather}
\end{document}